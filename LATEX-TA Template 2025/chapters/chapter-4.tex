\newpage
\chapter{HASIL DAN PEMBAHASAN} \label{Bab IV}

\section{Penggunaan Dataset} \label{IV.Penggunaan}
Dataset yang digunakan pada penelitian ini merupakan dataset PURE. 
\lipsum[1-2] % Menampilkan paragraf 1 sampai 2 dari lorem ipsum
% Atur margin untuk landscape - bisa disesuaikan nilai left dan right
\newgeometry{top=2cm, bottom=1.75cm, left=2cm, right=2cm}

\begin{landscape}
% \thispagestyle{landscape} % Untuk page number yang benar di landscape
\renewcommand{\arraystretch}{1.2} % Mengatur jarak antar baris tabel
\centering 
\begin{longtable}{|c|c|c|c|c|c|c|c|c|c|c|c|c|}
    \caption{Hasil Pengujian Model}
    \label{table:4.Pengujian1}\\
    \hline
    \multirow{2}{*}{\parbox{0.375cm}{\centering\textbf{No}}} & 
    \multirow{2}{*}{\parbox{4.0cm}{\centering\textbf{Citra}}} & 
    \multirow{2}{*}{\parbox{0.75cm}{\centering\textbf{GT\\Total}}} & 
    \multicolumn{8}{c|}{\textbf{Prediksi}} &
    \multirow{2}{*}{\parbox{1.2cm}{\centering\textbf{Hitung\\Manual}}} &
    \multirow{2}{*}{\parbox{1cm}{\centering\textbf{Model}}} \\
    \cline{4-11}
    & & &
    \parbox{0.75cm}{\centering\textbf{Total}} &
    \parbox{1.1cm}{\centering\textbf{Correct}} &
    \parbox{0.7cm}{\centering\textbf{\shortstack{Low\\[-0.2em]Conf}}} &
    \parbox{0.95cm}{\centering\textbf{\shortstack{Wrong\\[-0.35em]Loc}}} &
    \parbox{0.7cm}{\centering\textbf{\shortstack{False\\[-0.2em]Pos}}} &
    \parbox{0.7cm}{\centering\textbf{\shortstack{False\\[-0.2em]Neg}}} &
    \parbox{0.95cm}{\centering\textbf{\shortstack{AP\\[-0.2em]50:95}}} &
    \parbox{0.95cm}{\centering\textbf{\shortstack{IoU\\[-0.2em]50:95}}} &
    & \\

    \hline
    \endfirsthead
    
    \hline
    \multirow{2}{*}{\parbox{0.375cm}{\centering\textbf{No}}} & 
    \multirow{2}{*}{\parbox{4.0cm}{\centering\textbf{Citra}}} & 
    \multirow{2}{*}{\parbox{0.75cm}{\centering\textbf{GT\\Total}}} & 
    \multicolumn{8}{c|}{\textbf{Prediksi}} &
    \multirow{2}{*}{\parbox{1.2cm}{\centering\textbf{Hitung\\Manual}}} &
    \multirow{2}{*}{\parbox{1cm}{\centering\textbf{Model}}} \\
    \cline{4-11}
    & & &
    \parbox{0.75cm}{\centering\textbf{Total}} &
    \parbox{1.1cm}{\centering\textbf{Correct}} &
    \parbox{0.7cm}{\centering\textbf{\shortstack{Low\\[-0.2em]Conf}}} &
    \parbox{0.95cm}{\centering\textbf{\shortstack{Wrong\\[-0.35em]Loc}}} &
    \parbox{0.7cm}{\centering\textbf{\shortstack{False\\[-0.2em]Pos}}} &
    \parbox{0.7cm}{\centering\textbf{\shortstack{False\\[-0.2em]Neg}}} &
    \parbox{0.95cm}{\centering\textbf{\shortstack{AP\\[-0.2em]50:95}}} &
    \parbox{0.95cm}{\centering\textbf{\shortstack{IoU\\[-0.2em]50:95}}} &
    & \\
    \hline
    \endhead
    
    \hline
    \endfoot
    
    \hline
    \endlastfoot
    
    % ===== Data ke-1 =====
    \multirow{2}{*}{1} &
    \multirow{2}{*}{%
    \parbox[c][1.5cm][c]{4cm}{%
    \centering
    \includegraphics[width=4.05cm,height=4cm,keepaspectratio]{figure/CLIP.png}}} &
    \multirow{2}{*}{18} &

    34\rule{0pt}{3.0em} & 11 & 0 & 1 & 22 & 6 & 0{,}567 & 0{,}938 & 11 &
    V1 \textit{(Am)} \\
    \cline{4-13}

    & & &   % ← WAJIB: kosongkan No | Citra | GT Total
    36\rule{0pt}{3.0em} & 10 & 1 & 1 & 24 & 6 & 0{,}552 & 0{,}916 & 11 &
    V2 \textit{(Std)} \\

    \hline
    
    % ===== Data ke-2 =====
    \multirow{2}{*}{2} &
    \multirow{2}{*}{%
    \parbox[c][1.5cm][c]{4cm}{%
    \centering
    \includegraphics[width=4.05cm,height=4cm,keepaspectratio]{figure/CLIP.png}}} &
    \multirow{2}{*}{8} &

    16\rule{0pt}{3.0em} & 5 & 1 & 0 & 10 & 2 & 0{,}675 & 0{,}940 & 6 &
    V1 \textit{(Am)} \\

    \cline{4-13}

    & & &   % ← WAJIB: kosongkan No | Citra | GT Total
    28\rule{0pt}{3.0em} & 7 & 1 & 0 & 20 & 0 & 0{,}838 & 0{,}920 & 8 &
    V2 \textit{(Std)} \\

    \hline

    % ===== Data ke-3 =====
    \multirow{2}{*}{3} &
    \multirow{2}{*}{%
    \parbox[c][1.5cm][c]{4cm}{%
    \centering
    \includegraphics[width=4.05cm,height=4cm,keepaspectratio]{figure/CLIP.png}}} &
    \multirow{2}{*}{10} &

    21\rule{0pt}{3.0em} & 7 & 2 & 0 & 12 & 1 & 0{,}830 & 0{,}936 & 9 &
    V1 \textit{(Am)} \\

    \cline{4-13}

    & & &   % ← WAJIB: kosongkan No | Citra | GT Total
    19\rule{0pt}{3.0em} & 6 & 3 & 0 & 10 & 1 & 0{,}820 & 0{,}931 & 9 &
    V2 \textit{(Std)} \\

    \hline
\end{longtable}
\end{landscape}
\restoregeometry

\section{Akuisisi Gambar} \label{IV.Akuisisi}
Pada tahap ini, proses pembacaan dataset dilakukan dengan seksama untuk memastikan setiap gambar diperoleh dengan urutan yang benar dan sistematis. Penting untuk memastikan bahwa gambar yang diperoleh terurut dalam format \textit{time-series} agar memudahkan analisis pergerakan wajah yang terjadi dalam video. Implementasi kode yang digunakan untuk proses ini dapat dilihat pada \cref{code:4.akuisisi}. \par

\begin{lstlisting}[caption={Akuisisi Gambar}, label={code:4.akuisisi}, basicstyle=\ttfamily\scriptsize]
DATASET_ROOT = os.path.join(os.getcwd(), 'PURE Dataset')
def get_all_dataset_folders(root_path):
    dataset_folders = []
    for root, dirs, files in os.walk(root_path):
        if any(file.endswith('.png') for file in files):
            dataset_folders.append(root)
    return dataset_folder
def process_dataset(dataset_path):
    image_files = glob(os.path.join(dataset_path, '*.png'))
    image_files.sort()
    for image_file in image_files:
        frame = cv2.imread(image_file)
        if frame is None:
            continue
        frame_rgb = cv2.cvtColor(frame, cv2.COLOR_BGR2RGB)
        cv2.imshow('Frame', frame)
        if cv2.waitKey(1) & 0xFF == ord('q'):
            break
    cv2.destroyAllWindows()
def main():
    datasets = get_all_dataset_folders(DATASET_ROOT)
    for dataset in datasets:
        process_dataset(dataset)
\end{lstlisting}


\cref{code:4.akuisisi} merupakan baris kode untuk melakukan akuisisi gambar. 


\section{Analisis Hasil Penelitian} \label{IV.Analisis}
\lipsum[1-2] % Menampilkan paragraf 1 sampai 2 dari lorem ipsum


\section{Pembahasan} \label{IV.Bahas}
\lipsum[1-2] % Menampilkan paragraf 1 sampai 2 dari lorem ipsum